%% Created by Maple 2024.2, Linux
%% Source Worksheet: MapleExercises
%% Generated: Tue Dec 10 14:17:55 CET 2024
\documentclass{article}
\usepackage{amssymb}
\usepackage{graphicx}
\usepackage{hyperref}
\usepackage{listings}
\usepackage{mathtools}
\usepackage{maple}
\usepackage[utf8]{inputenc}
\usepackage[svgnames]{xcolor}
\usepackage{amsmath}
\usepackage{breqn}
\usepackage{textcomp}
\begin{document}
\lstset{basicstyle=\ttfamily,breaklines=true,columns=flexible}
\pagestyle{empty}
\DefineParaStyle{Maple Bullet Item}
\DefineParaStyle{Maple Heading 1}
\DefineParaStyle{Maple Warning}
\DefineParaStyle{Maple Heading 4}
\DefineParaStyle{Maple Heading 2}
\DefineParaStyle{Maple Heading 3}
\DefineParaStyle{Maple Dash Item}
\DefineParaStyle{Maple Error}
\DefineParaStyle{Maple Title}
\DefineParaStyle{Maple Ordered List 1}
\DefineParaStyle{Maple Text Output}
\DefineParaStyle{Maple Ordered List 2}
\DefineParaStyle{Maple Ordered List 3}
\DefineParaStyle{Maple Normal}
\DefineParaStyle{Maple Ordered List 4}
\DefineParaStyle{Maple Ordered List 5}
\DefineCharStyle{Maple 2D Output}
\DefineCharStyle{Maple 2D Input}
\DefineCharStyle{Maple Maple Input}
\DefineCharStyle{Maple 2D Math}
\DefineCharStyle{Maple Hyperlink}
\begin{Maple Normal}
Sanity cheking home exercise 1 results:
\end{Maple Normal}
\begin{Maple Normal}

\end{Maple Normal}
\begin{Maple Normal}
{$ \displaystyle x^{4}+x^{2}+1=\mathit{Factor} (x^{4}+x^{2}+1)\boldsymbol{\mod}2;
\\
 x^{3}+x^{1}+1=\mathit{Factor} (x^{3}+x^{1}+1)\boldsymbol{\mod}3; $}
\end{Maple Normal}
% \mapleresult
\begin{dmath*}
x^{4}+x^{2}+1=(x^{2}+x +1)^{2}
\end{dmath*}
\vspace{-\bigskipamount}
% \mapleresult
\begin{dmath}\label{(1)}
x^{3}+x +1=\left(x^{2}+x +2\right) \left(x +2\right)
\end{dmath}
\begin{Maple Normal}

\end{Maple Normal}
\begin{Maple Normal}

\end{Maple Normal}
\begin{Maple Normal}

\end{Maple Normal}
\begin{Maple Normal}
Factoring polynomials in Lab Exercise 1, if no factors are found then polynomial is reducible:
\end{Maple Normal}
\begin{Maple Normal}

\end{Maple Normal}
\begin{Maple Normal}
{$ \displaystyle \mathit{p1} \coloneqq x^{23}+x^{5}+1\colon \mathit{
\\
\,}\mathit{p2} \coloneqq x^{23}+x^{6}+1\colon \mathit{
\\
\,}x^{23}+x^{5}+1=\mathit{Factor} \mathit{(\mathit{p1})}\boldsymbol{\mod}2;\mathit{
\\
\,}x^{23}+x^{6}+1=\mathit{Factor} \mathit{(\mathit{p2})}\boldsymbol{\mod}2;\mathit{
\\
\,}\mathit{f2} \coloneqq \mathit{(\mathit{Factors}\,\mathit{(\mathit{p2})}\boldsymbol{\mod}2)\,}\mathit{[2]}\colon \mathit{
\\
\,}x^{18}+x^{3}+1=\mathit{Factor} \mathit{(x^{18}+x^{3}+1)}\boldsymbol{\mod}2;\mathit{
\\
\,}x^{8}+x^{6}+1=\mathit{Factor} \mathit{(x^{8}+x^{6}+1)}\boldsymbol{\mod}7; $}
\end{Maple Normal}
% \mapleresult
\begin{dmath*}
x^{23}+x^{5}+1=x^{23}+x^{5}+1
\end{dmath*}
\vspace{-\bigskipamount}
% \mapleresult
\begin{dmath*}
x^{23}+x^{6}+1=(x^{16}+x^{15}+x^{13}+x^{12}+x^{8}+x^{6}+x^{4}+x^{3}+x^{2}+x +1) (x^{3}+x +1) (x^{4}+x^{3}+1)
\end{dmath*}
\vspace{-\bigskipamount}
% \mapleresult
\begin{dmath*}
x^{18}+x^{3}+1=x^{18}+x^{3}+1
\end{dmath*}
\vspace{-\bigskipamount}
% \mapleresult
\begin{dmath}\label{(2)}
x^{8}+x^{6}+1=\left(x^{4}+3 x^{3}+5 x^{2}+5 x +6\right) \left(x^{4}+4 x^{3}+5 x^{2}+2 x +6\right)
\end{dmath}
\begin{Maple Normal}

\end{Maple Normal}
\begin{Maple Normal}

\end{Maple Normal}
\begin{Maple Normal}
Setting up the element \alpha and calculating order of \alpha, \alpha^2, \alpha^3, \alpha + \alpha^3:
\end{Maple Normal}
\begin{Maple Normal}

\end{Maple Normal}
\begin{Maple Normal}
{$ \displaystyle \mathit{G218} \coloneqq \mathit{GF} (2,18,\alpha^{18}+\alpha^{3}+1)\colon 
\\
 a \coloneqq \mathit{G218} (\alpha)\colon 
\\
 \mathit{ord} (\alpha)=\mathit{G218} \mcoloneq \mathit{order} (a);
\\
 \mathit{ord} (\alpha^{2})=\mathit{G218} \mcoloneq \mathit{order} (\mathit{G218} \colon -\mathit{{\mhat}}(a ,2));
\\
 \mathit{ord} (\alpha^{3})=\mathit{G218} \mcoloneq \mathit{order} (\mathit{G218} \colon -\mathit{{\mhat}}(a ,3));
\\
 \mathit{ord} (\alpha +\alpha^{3})=\mathit{G218} \mcoloneq \mathit{order} (\mathit{G218} \mcoloneq \mathit{+}(a ,\mathit{G218} \colon -\mathit{{\mhat}}(a ,3))); $}
\end{Maple Normal}
% \mapleresult
\begin{dmath*}
\mathit{ord} \! \left(\alpha \right)=189
\end{dmath*}
\vspace{-\bigskipamount}
% \mapleresult
\begin{dmath*}
\mathit{ord} \! \left(\alpha^{2}\right)=189
\end{dmath*}
\vspace{-\bigskipamount}
% \mapleresult
\begin{dmath*}
\mathit{ord} \! \left(\alpha^{3}\right)=63
\end{dmath*}
\vspace{-\bigskipamount}
% \mapleresult
\begin{dmath}\label{(3)}
\mathit{ord} \! \left(\alpha^{3}+\alpha \right)=262143
\end{dmath}
\begin{Maple Normal}

\end{Maple Normal}
\begin{Maple Normal}

\end{Maple Normal}
\begin{Maple Normal}
Lab exercise 3:
\end{Maple Normal}
\begin{Maple Normal}
{$ \displaystyle \mathit{Primitive} \mathit{(\mathit{p1})}\boldsymbol{\mod}2 $}
\end{Maple Normal}
% \mapleresult
\begin{dmath}\label{(4)}
\mathit{true} 
\end{dmath}
\begin{Maple Normal}
Since p1 is primitive, it follows that\

the cycle set for p1 = 1(1) + 1(2^23-1).
\end{Maple Normal}
\begin{Maple Normal}

\end{Maple Normal}
\begin{Maple Normal}

\end{Maple Normal}
\begin{Maple Normal}

\end{Maple Normal}
\begin{Maple Normal}
For the second polynomial we have:
\end{Maple Normal}
\begin{Maple Normal}
{$ \displaystyle \mathit{Primitive} (\mathit{f2} [1][1])\boldsymbol{\mod}2;
\\
 \mathit{Primitive} (\mathit{f2} [2][1])\boldsymbol{\mod}2;
\\
 \mathit{Primitive} (\mathit{f2} [3][1])\boldsymbol{\mod}2; $}
\end{Maple Normal}
% \mapleresult
\begin{dmath*}
\mathit{false} 
\end{dmath*}
\vspace{-\bigskipamount}
% \mapleresult
\begin{dmath*}
\mathit{true} 
\end{dmath*}
\vspace{-\bigskipamount}
% \mapleresult
\begin{dmath}\label{(5)}
\mathit{true} 
\end{dmath}
\begin{Maple Normal}
{$ \displaystyle \mathit{T1} =2^{\mathit{degree} }(\mathit{f2} [2][1])-1;
\\
 \mathit{T3} =2^{\mathit{degree} }(\mathit{f2} [3][1])-1; $}
\end{Maple Normal}
% \mapleresult
\begin{dmath*}
\mathit{T1} =7
\end{dmath*}
\vspace{-\bigskipamount}
% \mapleresult
\begin{dmath}\label{(6)}
\mathit{T3} =15
\end{dmath}
\begin{Maple Normal}
{$ \displaystyle f \coloneqq \mathit{f2} [1][1]\colon 
\\
 \mathit{divs} \coloneqq \mathit{NumberTheory} \mcoloneq \mathit{Divisors} (2^{\mathit{degree} (f)}-1)\colon 
\\
 i \coloneqq 1\colon 
\\
 n \coloneqq \mathit{factor} (\mathit{degree} (f)-1)\colon 
\\
 
\\
 \boldsymbol{\mathrm{while}}\boldsymbol{\neg}(\mathit{Divide} (x^{\mathit{divs} [i]}+1,f)\boldsymbol{\mod}2)\boldsymbol{\land}i <2^{\mathit{degree} \mathit{(f)}}\boldsymbol{\mathrm{do}}
\\
 i ++\colon 
\\
 \boldsymbol{\mathrm{end}}\boldsymbol{\mathrm{do}}\colon 
\\
 \mathit{T2} =\mathit{divs} [i] $}
\end{Maple Normal}
% \mapleresult
\begin{dmath}\label{(7)}
\mathit{T2} =21845
\end{dmath}
\begin{Maple Normal}
All in all we have: C1 = [1(1) + 1(15)], C2 = [1(1) + 3 (21845)], C3 = [1(1) + 1(7)]
\end{Maple Normal}
\begin{Maple Normal}
{$ \displaystyle \mathit{C12} \coloneqq [1,1],[1,15],[3,\mathit{divs} [i]],[3\cdot \gcd (15,\mathit{divs} [i]),\mathit{lcm} (15,\mathit{divs} [i])] $}
\end{Maple Normal}
% \mapleresult
\begin{dmath}\label{(8)}
\mathit{C12} \coloneqq \left[1,1\right],\left[1,15\right],\left[3,21845\right],\left[15,65535\right]
\end{dmath}
\begin{Maple Normal}
{$ \displaystyle \mathit{C123} \coloneqq \mathit{C12} ,[1\cdot \gcd (7,15),\mathit{lcm} (7,15)],[3\cdot \gcd (7,\mathit{divs} [i]),\mathit{lcm} (7,\mathit{divs} [i])],[15\cdot \gcd (7,65535),\mathit{lcm} (7,65535)] $}
\end{Maple Normal}
% \mapleresult
\begin{dmath}\label{(9)}
\mathit{C123} \coloneqq \left[1,1\right],\left[1,15\right],\left[3,21845\right],\left[15,65535\right],\left[1,105\right],
\\
\left[3,152915\right],\left[15,458745\right]
\end{dmath}
\begin{Maple Normal}

\end{Maple Normal}
\begin{Maple Normal}
{$ \displaystyle z \coloneqq 4\colon 
\\
 \boldsymbol{\mathrm{for}}i \boldsymbol{\mathrm{from}}1\boldsymbol{\mathrm{to}}z \boldsymbol{\mathrm{do}}
\\
 \boldsymbol{\mathrm{for}}j \boldsymbol{\mathrm{from}}0\boldsymbol{\mathrm{to}}z \boldsymbol{\mathrm{do}}
\\
 \boldsymbol{\mathrm{for}}k \boldsymbol{\mathrm{from}}0\boldsymbol{\mathrm{to}}z \boldsymbol{\mathrm{do}}
\\
 \boldsymbol{\mathrm{for}}l \boldsymbol{\mathrm{from}}0\boldsymbol{\mathrm{to}}z \boldsymbol{\mathrm{do}}
\\
 
\\
 \boldsymbol{\mathrm{if}}(\mathit{Primitive} (i \cdot x^{4}+j \cdot x^{3}+k \cdot x^{2}+l \cdot x +1)\boldsymbol{\mod}5)\boldsymbol{\mathrm{then}}
\\
 \mathit{print} (i \cdot x^{4}+j \cdot x^{3}+k \cdot x^{2}+l \cdot x +1);
\\
 \esnum \mathit{break\,i;\,} 
\\
 \boldsymbol{\mathrm{end}}\boldsymbol{\mathrm{if}};
\\
 \boldsymbol{\mathrm{end}}\boldsymbol{\mathrm{do}};
\\
 \boldsymbol{\mathrm{end}}\boldsymbol{\mathrm{do}};
\\
 
\\
 \boldsymbol{\mathrm{end}}\boldsymbol{\mathrm{do}};
\\
 \boldsymbol{\mathrm{end}}\boldsymbol{\mathrm{do}}; $}
\end{Maple Normal}
% \mapleresult
\begin{dmath*}
2 x^{4}+2 x^{2}+x +1
\end{dmath*}
\vspace{-\bigskipamount}
% \mapleresult
\begin{dmath*}
2 x^{4}+2 x^{2}+4 x +1
\end{dmath*}
\vspace{-\bigskipamount}
% \mapleresult
\begin{dmath*}
2 x^{4}+3 x^{2}+2 x +1
\end{dmath*}
\vspace{-\bigskipamount}
% \mapleresult
\begin{dmath*}
2 x^{4}+3 x^{2}+3 x +1
\end{dmath*}
\vspace{-\bigskipamount}
% \mapleresult
\begin{dmath*}
2 x^{4}+x^{3}+3 x +1
\end{dmath*}
\vspace{-\bigskipamount}
% \mapleresult
\begin{dmath*}
2 x^{4}+x^{3}+4 x +1
\end{dmath*}
\vspace{-\bigskipamount}
% \mapleresult
\begin{dmath*}
2 x^{4}+x^{3}+2 x^{2}+x +1
\end{dmath*}
\vspace{-\bigskipamount}
% \mapleresult
\begin{dmath*}
2 x^{4}+x^{3}+3 x^{2}+4 x +1
\end{dmath*}
\vspace{-\bigskipamount}
% \mapleresult
\begin{dmath*}
2 x^{4}+x^{3}+4 x^{2}+1
\end{dmath*}
\vspace{-\bigskipamount}
% \mapleresult
\begin{dmath*}
2 x^{4}+2 x^{3}+2 x +1
\end{dmath*}
\vspace{-\bigskipamount}
% \mapleresult
\begin{dmath*}
2 x^{4}+2 x^{3}+4 x +1
\end{dmath*}
\vspace{-\bigskipamount}
% \mapleresult
\begin{dmath*}
2 x^{4}+2 x^{3}+x^{2}+1
\end{dmath*}
\vspace{-\bigskipamount}
% \mapleresult
\begin{dmath*}
2 x^{4}+2 x^{3}+2 x^{2}+2 x +1
\end{dmath*}
\vspace{-\bigskipamount}
% \mapleresult
\begin{dmath*}
2 x^{4}+2 x^{3}+3 x^{2}+3 x +1
\end{dmath*}
\vspace{-\bigskipamount}
% \mapleresult
\begin{dmath*}
2 x^{4}+3 x^{3}+x +1
\end{dmath*}
\vspace{-\bigskipamount}
% \mapleresult
\begin{dmath*}
2 x^{4}+3 x^{3}+3 x +1
\end{dmath*}
\vspace{-\bigskipamount}
% \mapleresult
\begin{dmath*}
2 x^{4}+3 x^{3}+x^{2}+1
\end{dmath*}
\vspace{-\bigskipamount}
% \mapleresult
\begin{dmath*}
2 x^{4}+3 x^{3}+2 x^{2}+3 x +1
\end{dmath*}
\vspace{-\bigskipamount}
% \mapleresult
\begin{dmath*}
2 x^{4}+3 x^{3}+3 x^{2}+2 x +1
\end{dmath*}
\vspace{-\bigskipamount}
% \mapleresult
\begin{dmath*}
2 x^{4}+4 x^{3}+x +1
\end{dmath*}
\vspace{-\bigskipamount}
% \mapleresult
\begin{dmath*}
2 x^{4}+4 x^{3}+2 x +1
\end{dmath*}
\vspace{-\bigskipamount}
% \mapleresult
\begin{dmath*}
2 x^{4}+4 x^{3}+2 x^{2}+4 x +1
\end{dmath*}
\vspace{-\bigskipamount}
% \mapleresult
\begin{dmath*}
2 x^{4}+4 x^{3}+3 x^{2}+x +1
\end{dmath*}
\vspace{-\bigskipamount}
% \mapleresult
\begin{dmath*}
2 x^{4}+4 x^{3}+4 x^{2}+1
\end{dmath*}
\vspace{-\bigskipamount}
% \mapleresult
\begin{dmath*}
3 x^{4}+2 x^{2}+2 x +1
\end{dmath*}
\vspace{-\bigskipamount}
% \mapleresult
\begin{dmath*}
3 x^{4}+2 x^{2}+3 x +1
\end{dmath*}
\vspace{-\bigskipamount}
% \mapleresult
\begin{dmath*}
3 x^{4}+3 x^{2}+x +1
\end{dmath*}
\vspace{-\bigskipamount}
% \mapleresult
\begin{dmath*}
3 x^{4}+3 x^{2}+4 x +1
\end{dmath*}
\vspace{-\bigskipamount}
% \mapleresult
\begin{dmath*}
3 x^{4}+x^{3}+x +1
\end{dmath*}
\vspace{-\bigskipamount}
% \mapleresult
\begin{dmath*}
3 x^{4}+x^{3}+2 x +1
\end{dmath*}
\vspace{-\bigskipamount}
% \mapleresult
\begin{dmath*}
3 x^{4}+x^{3}+x^{2}+x +1
\end{dmath*}
\vspace{-\bigskipamount}
% \mapleresult
\begin{dmath*}
3 x^{4}+x^{3}+4 x^{2}+1
\end{dmath*}
\vspace{-\bigskipamount}
% \mapleresult
\begin{dmath*}
3 x^{4}+x^{3}+4 x^{2}+4 x +1
\end{dmath*}
\vspace{-\bigskipamount}
% \mapleresult
\begin{dmath*}
3 x^{4}+2 x^{3}+x +1
\end{dmath*}
\vspace{-\bigskipamount}
% \mapleresult
\begin{dmath*}
3 x^{4}+2 x^{3}+3 x +1
\end{dmath*}
\vspace{-\bigskipamount}
% \mapleresult
\begin{dmath*}
3 x^{4}+2 x^{3}+x^{2}+1
\end{dmath*}
\vspace{-\bigskipamount}
% \mapleresult
\begin{dmath*}
3 x^{4}+2 x^{3}+x^{2}+2 x +1
\end{dmath*}
\vspace{-\bigskipamount}
% \mapleresult
\begin{dmath*}
3 x^{4}+2 x^{3}+4 x^{2}+3 x +1
\end{dmath*}
\vspace{-\bigskipamount}
% \mapleresult
\begin{dmath*}
3 x^{4}+3 x^{3}+2 x +1
\end{dmath*}
\vspace{-\bigskipamount}
% \mapleresult
\begin{dmath*}
3 x^{4}+3 x^{3}+4 x +1
\end{dmath*}
\vspace{-\bigskipamount}
% \mapleresult
\begin{dmath*}
3 x^{4}+3 x^{3}+x^{2}+1
\end{dmath*}
\vspace{-\bigskipamount}
% \mapleresult
\begin{dmath*}
3 x^{4}+3 x^{3}+x^{2}+3 x +1
\end{dmath*}
\vspace{-\bigskipamount}
% \mapleresult
\begin{dmath*}
3 x^{4}+3 x^{3}+4 x^{2}+2 x +1
\end{dmath*}
\vspace{-\bigskipamount}
% \mapleresult
\begin{dmath*}
3 x^{4}+4 x^{3}+3 x +1
\end{dmath*}
\vspace{-\bigskipamount}
% \mapleresult
\begin{dmath*}
3 x^{4}+4 x^{3}+4 x +1
\end{dmath*}
\vspace{-\bigskipamount}
% \mapleresult
\begin{dmath*}
3 x^{4}+4 x^{3}+x^{2}+4 x +1
\end{dmath*}
\vspace{-\bigskipamount}
% \mapleresult
\begin{dmath*}
3 x^{4}+4 x^{3}+4 x^{2}+1
\end{dmath*}
\vspace{-\bigskipamount}
% \mapleresult
\begin{dmath}\label{(10)}
3 x^{4}+4 x^{3}+4 x^{2}+x +1
\end{dmath}
\begin{Maple Normal}
{$ \displaystyle \mathit{Primitive} (2\cdot x^{4}+2\cdot x^{2}+x +1)\boldsymbol{\mod}5 $}
\end{Maple Normal}
% \mapleresult
\begin{dmath}\label{(11)}
\mathit{true} 
\end{dmath}
\begin{Maple Normal}

\end{Maple Normal}
\end{document}
